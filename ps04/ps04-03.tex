\documentclass{article}
\usepackage{amsmath, amsfonts, amssymb, amsthm}
\usepackage{enumitem, tikz, mathtools}

\title{PS04-03}
\date{\today}

\begin{document}
\maketitle
Consider grammar $G$:
\begin{align*}
    S &\rightarrow ABS | AB\\
    A &\rightarrow aA | a\\
    B &\rightarrow bA.
\end{align*}

\begin{enumerate}[label=\alph*.]
    \item Are the following strings in $L(G)$?
    \begin{enumerate}[label=\roman*.]
        \item \textit{aabaab} $\notin L(G)$. This is because the only terminal in $G$ is $a$.
        \item \textit{aaaaba} $S \rightarrow AB \rightarrow aAB \rightarrow aaAB \rightarrow aaaAB \rightarrow aaaaB \rightarrow aaaabA \rightarrow aaaaba$.
        \item \textit{aabbaa} $\notin L(G)$. This is because a $b$ is always
        followed by an $a$ (rule $S$). If there are two $b$s in a string in
        this language, there are always at least 2 $a$s in between.
        \item \textit{abaaba} $S \rightarrow ABS \rightarrow aBS \rightarrow abAS \rightarrow abaS \rightarrow abaAB \rightarrow abaaB \rightarrow abaabA \rightarrow abaaba$.
    \end{enumerate}
    \item $G = \langle V, \Sigma, R, S \rangle$,\\
    $L(G) = \{w \in \Sigma^* : S \xRightarrow[]{\text{*}} w\}$\\
    $L(G) = \{a^{i_1}ba^{j_1} \cdot... \cdot a^{i_n}ba^{j_n} : i,j,n > 0\}$ %do this part! ask gales
\end{enumerate}

\end{document}
