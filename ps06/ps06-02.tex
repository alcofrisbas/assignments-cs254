\documentclass{article}
\usepackage{amsmath, amsfonts, amssymb, amsthm}
\usepackage{enumitem, tikz}

\title{PS06-02}
\date{\today}

\begin{document}
\maketitle
\begin{enumerate}[label=\alph*]
\item Intersection.\\
Much like union. Given two languages $A$ and $B$, let $M_A$ and $M_B$ be two Turing machines s.t. $L(M_A) = A$ and $L(M_B) = B$.\\
Run both simultaneously on two tracks, and if both accept, then accept
\item Concatenate.\\
Given two languages $A$ and $B$, let $M_A$ and $M_B$ be two Turing machines s.t. $L(M_A) = A$ and $L(M_B) = B$.\\
Let $w$ be an input string. $w = w_1\circ w_2$
Let $\lvert w_1 \rvert = 1$ and $\lvert w_2 \rvert = \lvert w \rvert - \lvert w_1 \rvert$
Run $M_A$ on $w_1$ and $M_B$ on $w_2$. If both accept, accept, otherwise add 1 to $\lvert w_1 \rvert$ and try again. Keep trying, increasing $\lvert w_1 \rvert$ until accept or until the length of $w_1$ equals the length of $w_2$, at which point, reject.
\item *\\
This one is computationally intensive. Start the same way as concatenate. Essentially, run $M_A$ on every possible set of substrings of the input string, and if at any point, all the strings in the set accept, accept.
\item In order to prove closure for recursive languages, you need to prove decidability for the constituent languges. In the event of $A$ not being a recursive language(just r.e.), intersection could potentially never reject, just accept and loop forever. 
\end{enumerate}
\end{document}
