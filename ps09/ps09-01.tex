\documentclass{article}
\usepackage{amsmath, amsfonts, amssymb, amsthm}
\usepackage{enumitem, tikz}

\title{PS09-01}
\date{\today}

\begin{document}
\maketitle
Make an NTM $M$. On input $\langle G, H \rangle$:\\
\begin{enumerate}
	\item Verify that $\lvert V \rvert = \lvert V' \rvert$ and $\lvert E\rvert = \lvert E' \rvert$.
	\item If that is not the case, then there is no way $G$ and $H$ are \textit{isomorphic}. Reject.
	\item We know that in order for the graphs to be isomorphic, there needs to exist a bejective function
	that maps $V$ to $V'$. Essentially, we will ``guess'' by mapping permutations of $V$ to $V'$, then check
	by seeing if all of the $f(u)$s and $f(v)$s behave in $E'$ like they do in $E$. If so, accept. If not, then
	make another guess. If there are no more guesses, reject.
\end{enumerate}
There are three components of this algorithm. The verifying, the guessing, and the checking.\\
The verification process involves summing and comparing the lengths of the sets, making it a linear time process.\\
Guessing involves translating value-by-value with a bit of overhead per value, which is also a linear time process.\\
The checking process is also linear because it involves direct, one-to-one comparison.\\ 
In the worst case scenario, the machine will have to make ${\lvert V \rvert \choose 2}$ guesses. 
\begin{center}
\line(1,0){250}
\end{center}
\subsubsection*{Lemma: n choose 2 is $O(n^2)$}

\begin{align*}
{n \choose 2} &= \frac{n!}{(n-2)!2!}&&\\
&= \frac{n(n-1)}{2!}&&\\
&= n^2 && \text{disregard constants}\\
\end{align*}
little box.
\begin{center}
\line(1,0){250}
\end{center}

Which is a $O(n^2)$ operation. Therefore, this machine runs in polynomial time, which means the \textsc{GraphIso} in in NP.\\
Box.
\end{document}
