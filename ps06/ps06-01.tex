\documentclass{article}
\usepackage{amsmath, amsfonts, amssymb, amsthm}
\usepackage{enumitem, tikz}

\title{PS06-01}
\date{\today}

\begin{document}
\maketitle
Regular Languages can be recognized by Turing Machines that only go right or stay put. 

Given a TM $M = \langle Q, \Sigma, \Gamma, \vdash, \_, \delta, s, t, r\rangle$, let $N$ be an NFA that recognizes the same language as $M$. Let $N = \langle Q_N, \Sigma_N, \delta_N, s_N, F_N \rangle$.\\
Let $Q_N = Q$.\\
Let $\delta_N(q_i,a) = q_j \text{ if } \delta(q_i, a) = \langle q_j, b, R \rangle$\\
and $\delta_N(q_i,\varepsilon) = q_j \text{ if } \delta(q_i, a) = \langle q_j, b, S \rangle$\\
Let $s_N = s$\\
Let $t \in F_N \text{ and } r \notin F_N$

Since there is no way for the machine to look back on what it wrote, it loses its memory capability, which means that it cannot recognize context free languages;in other words, the head is only a read head. Essentially, moving right is equivalent to consuming a character and staying put is equivalent to an epsilon transition. Therefore, any situation where the TM would stay put is Trivial.
\end{document}
